\documentclass[12pt]{article}

\usepackage{booktabs}
\usepackage[margin=0.75in]{geometry}
\usepackage{microtype}
\usepackage{tabulary}

\title{Integral types and classes}
\author{Eric Berquist}
\date{\today}

\begin{document}
\maketitle
\section{\texttt{Integral1e2c}}

\begin{tabulary}{1.00\linewidth}{lL}
  \toprule
  class name & requires \\
  \midrule
  \texttt{Overlap} & basis function (bf) centers/positions, bf exponents, bf contraction coefficients \\
  \texttt{Kinetic} \\
  \texttt{Potential} \\
  \texttt{FermiContact/FC} \\
  \texttt{SpinDipole/SD} \\
  \texttt{ElectricField/EF} \\
  \texttt{ElectricFieldGradient/EFG} \\
  \texttt{SpinOrbit} \\
  \texttt{Dipole/DipoleLength} \\
  \texttt{SecondMoment} \\
  \texttt{Quadrupole} \\
  \texttt{TracelessQuadrupole} \\
  \texttt{Multipole} \\
  \texttt{Nabla/DipoleVelocity} \\
  \texttt{AngularMomentum} \\
  \texttt{OverlapDer1} \\
  \texttt{KineticDer1} \\
  \texttt{PotentialDer1} \\
  \bottomrule
\end{tabulary}

\section{\texttt{Integral2e4c}}

\begin{tabulary}{1.00\linewidth}{lL}
  \toprule
  class name & requires \\
  \midrule
  \texttt{Repulsion} \\
  \texttt{SpinOrbitExact} \\
  \bottomrule
\end{tabulary}

\end{document}
